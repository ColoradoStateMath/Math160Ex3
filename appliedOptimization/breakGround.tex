\documentclass{ximera}

%\usepackage{todonotes}

\newcommand{\todo}{}

\usepackage{tkz-euclide}
\tikzset{>=stealth} %% cool arrow head
\tikzset{shorten <>/.style={ shorten >=#1, shorten <=#1 } } %% allows shorter vectors

\usetikzlibrary{backgrounds} %% for boxes around graphs
\usetikzlibrary{shapes,positioning}  %% Clouds and stars
\usetikzlibrary{matrix} %% for matrix
\usepgfplotslibrary{polar} %% for polar plots
\usetkzobj{all}
\usepackage[makeroom]{cancel} %% for strike outs
%\usepackage{mathtools} %% for pretty underbrace % Breaks Ximera
\usepackage{multicol}





\usepackage{array}
\setlength{\extrarowheight}{+.1cm}   
\newdimen\digitwidth
\settowidth\digitwidth{9}
\def\divrule#1#2{
\noalign{\moveright#1\digitwidth
\vbox{\hrule width#2\digitwidth}}}





\newcommand{\RR}{\mathbb R}
\newcommand{\R}{\mathbb R}
\newcommand{\N}{\mathbb N}
\newcommand{\Z}{\mathbb Z}

%\renewcommand{\d}{\,d\!}
\renewcommand{\d}{\mathop{}\!d}
\newcommand{\dd}[2][]{\frac{\d #1}{\d #2}}
\newcommand{\pp}[2][]{\frac{\partial #1}{\partial #2}}
\renewcommand{\l}{\ell}
\newcommand{\ddx}{\frac{d}{\d x}}

\newcommand{\zeroOverZero}{\ensuremath{\boldsymbol{\tfrac{0}{0}}}}
\newcommand{\inftyOverInfty}{\ensuremath{\boldsymbol{\tfrac{\infty}{\infty}}}}
\newcommand{\zeroOverInfty}{\ensuremath{\boldsymbol{\tfrac{0}{\infty}}}}
\newcommand{\zeroTimesInfty}{\ensuremath{\small\boldsymbol{0\cdot \infty}}}
\newcommand{\inftyMinusInfty}{\ensuremath{\small\boldsymbol{\infty - \infty}}}
\newcommand{\oneToInfty}{\ensuremath{\boldsymbol{1^\infty}}}
\newcommand{\zeroToZero}{\ensuremath{\boldsymbol{0^0}}}
\newcommand{\inftyToZero}{\ensuremath{\boldsymbol{\infty^0}}}



\newcommand{\numOverZero}{\ensuremath{\boldsymbol{\tfrac{\#}{0}}}}
\newcommand{\dfn}{\textbf}
%\newcommand{\unit}{\,\mathrm}
\newcommand{\unit}{\mathop{}\!\mathrm}
\newcommand{\eval}[1]{\bigg[ #1 \bigg]}
\newcommand{\seq}[1]{\left( #1 \right)}
\renewcommand{\epsilon}{\varepsilon}
\renewcommand{\iff}{\Leftrightarrow}

\DeclareMathOperator{\arccot}{arccot}
\DeclareMathOperator{\arcsec}{arcsec}
\DeclareMathOperator{\arccsc}{arccsc}
\DeclareMathOperator{\si}{Si}
\DeclareMathOperator{\proj}{proj}
\DeclareMathOperator{\scal}{scal}


\newcommand{\tightoverset}[2]{% for arrow vec
  \mathop{#2}\limits^{\vbox to -.5ex{\kern-0.75ex\hbox{$#1$}\vss}}}
\newcommand{\arrowvec}[1]{\tightoverset{\scriptstyle\rightharpoonup}{#1}}
\renewcommand{\vec}{\mathbf}
\newcommand{\veci}{\vec{i}}
\newcommand{\vecj}{\vec{j}}
\newcommand{\veck}{\vec{k}}
\newcommand{\vecl}{\boldsymbol{\l}}

\newcommand{\dotp}{\bullet}
\newcommand{\cross}{\boldsymbol\times}
\newcommand{\grad}{\boldsymbol\nabla}
\newcommand{\divergence}{\grad\dotp}
\newcommand{\curl}{\grad\cross}
%\DeclareMathOperator{\divergence}{divergence}
%\DeclareMathOperator{\curl}[1]{\grad\cross #1}


\colorlet{textColor}{black} 
\colorlet{background}{white}
\colorlet{penColor}{blue!50!black} % Color of a curve in a plot
\colorlet{penColor2}{red!50!black}% Color of a curve in a plot
\colorlet{penColor3}{red!50!blue} % Color of a curve in a plot
\colorlet{penColor4}{green!50!black} % Color of a curve in a plot
\colorlet{penColor5}{orange!80!black} % Color of a curve in a plot
\colorlet{fill1}{penColor!20} % Color of fill in a plot
\colorlet{fill2}{penColor2!20} % Color of fill in a plot
\colorlet{fillp}{fill1} % Color of positive area
\colorlet{filln}{penColor2!20} % Color of negative area
\colorlet{fill3}{penColor3!20} % Fill
\colorlet{fill4}{penColor4!20} % Fill
\colorlet{fill5}{penColor5!20} % Fill
\colorlet{gridColor}{gray!50} % Color of grid in a plot

\newcommand{\surfaceColor}{violet}
\newcommand{\surfaceColorTwo}{redyellow}
\newcommand{\sliceColor}{greenyellow}




\pgfmathdeclarefunction{gauss}{2}{% gives gaussian
  \pgfmathparse{1/(#2*sqrt(2*pi))*exp(-((x-#1)^2)/(2*#2^2))}%
}


%%%%%%%%%%%%%
%% Vectors
%%%%%%%%%%%%%

%% Simple horiz vectors
\renewcommand{\vector}[1]{\left\langle #1\right\rangle}


%% %% Complex Horiz Vectors with angle brackets
%% \makeatletter
%% \renewcommand{\vector}[2][ , ]{\left\langle%
%%   \def\nextitem{\def\nextitem{#1}}%
%%   \@for \el:=#2\do{\nextitem\el}\right\rangle%
%% }
%% \makeatother

%% %% Vertical Vectors
%% \def\vector#1{\begin{bmatrix}\vecListA#1,,\end{bmatrix}}
%% \def\vecListA#1,{\if,#1,\else #1\cr \expandafter \vecListA \fi}

%%%%%%%%%%%%%
%% End of vectors
%%%%%%%%%%%%%

%\newcommand{\fullwidth}{}
%\newcommand{\normalwidth}{}



%% makes a snazzy t-chart for evaluating functions
%\newenvironment{tchart}{\rowcolors{2}{}{background!90!textColor}\array}{\endarray}

%%This is to help with formatting on future title pages.
\newenvironment{sectionOutcomes}{}{} 



%% Flowchart stuff
%\tikzstyle{startstop} = [rectangle, rounded corners, minimum width=3cm, minimum height=1cm,text centered, draw=black]
%\tikzstyle{question} = [rectangle, minimum width=3cm, minimum height=1cm, text centered, draw=black]
%\tikzstyle{decision} = [trapezium, trapezium left angle=70, trapezium right angle=110, minimum width=3cm, minimum height=1cm, text centered, draw=black]
%\tikzstyle{question} = [rectangle, rounded corners, minimum width=3cm, minimum height=1cm,text centered, draw=black]
%\tikzstyle{process} = [rectangle, minimum width=3cm, minimum height=1cm, text centered, draw=black]
%\tikzstyle{decision} = [trapezium, trapezium left angle=70, trapezium right angle=110, minimum width=3cm, minimum height=1cm, text centered, draw=black]


\outcome{Interpret an optimization problem as the procedure used to
  make a system or design as effective or functional as possible.}

\outcome{Set up an optimization problem by identifying the objective
  function and appropriate constraints.}

\title[Break-Ground:]{Volumes of aluminum cans}

\begin{document}
\begin{abstract}
Two young mathematicians discuss optimizing aluminum cans.
\end{abstract}
\maketitle

Check out this dialogue between two calculus students (based on a true
story):

\begin{dialogue}
\item[Devyn] Riley, have you ever noticed aluminum cans?
\item[Riley] So very recyclable! 
\item[Devyn] I know! But I've also noticed that there are some that
  are short and fat, and others that are tall and skinny, and yet they
  can still have the same volume!
\item[Riley] So very observant! 
\item[Devyn] This got me wondering, if we want to make a can with
  volume $V$, what shape of can uses the least aluminum?
\item[Riley] Ah! This sounds like a job for calculus!  The volume of a cylindrical can is given by
  \[
  V = \pi \cdot r^2 \cdot h
  \]
  where $r$ is the radius of the cylinder and $h$ is the height of the
  cylinder. Also the surface area is given by
  \begin{align*}
    A &= \underbrace{\pi \cdot r^2}_{\text{bottom}} + \underbrace{2\cdot\pi \cdot r\cdot h}_{\text{sides}} + \underbrace{\pi \cdot r^2}_{\text{top}}\\
    &= 2\cdot \pi \cdot r^2 + 2\cdot\pi \cdot r\cdot h.    
  \end{align*}
  Somehow we want to minimize the surface area, because that's the
  amount of aluminum used, but we also want to keep the volume constant.
\item[Devyn] Whoa, we have way too many letters here.
\item[Riley] Yeah, somehow we need only one variable. Yikes. Too many letters.
\end{dialogue}

\begin{problem}
  Suppose we wish to construct an aluminum can with volume $V$ that
  uses the least amount of aluminum. In the context above, what do we
  want to minimize?
  \begin{multipleChoice}
    \choice[correct]{$A$}
    \choice{$V$}
    \choice{$h$}
    \choice{$r$}
  \end{multipleChoice}
\end{problem}

\begin{problem}
  In the context above, what should be considered a constant?
  \begin{selectAll}
    \choice{$A$}
    \choice[correct]{$V$}
    \choice{$h$}
    \choice{$r$}
  \end{selectAll}
\end{problem}

As Devyn and Riley noticed, when we work out this type of problem, 
we need to reduce the problem to a single variable.

\begin{problem}
  Consider $r$ to be the variable, and express $A$ as a function of $r$.
  \begin{hint}
    First, let's eliminate the variable $h$.  We know that $V$ is a constant and that
    \[
    V = \pi \cdot r^2 \cdot h
    \]
    so $h=\answer{V/(\pi r^2)}$.
    Substitute this expression for $h$ in the equation
    \[
  A = 2\cdot \pi \cdot r^2 + 2\cdot\pi \cdot r \cdot h.
    \]
  \end{hint}
  \begin{prompt}
    \[
    A = \answer{2\cdot \pi \cdot r^2 + 2\cdot\pi \cdot r\cdot V/(\pi r^2)}
    \]
  \end{prompt}
\end{problem}

\begin{problem}
  Now consider $h$ to be the variable, and express $A$ as a function of $h$.
  \begin{hint}
    First, let's eliminate the variable $r$.  We know that $V$ is a constant and that
    \[
    V = \pi \cdot r^2 \cdot h
    \]
    so $r=\answer{\sqrt{V/(\pi h)}}$.
    Substitute this expression for $r$ in the equation
    \[
    A = 2\cdot \pi \cdot r^2 + 2\cdot\pi \cdot r \cdot h.
    \]
  \end{hint}
  \begin{prompt}
    \[
    A = \answer{2\cdot \pi \cdot V/(\pi h) + 2\cdot\pi \cdot \sqrt{V/(\pi h)} \cdot h}
    \]
  \end{prompt}
\end{problem}

Notice that we've reduced (one way or another) this
function of two variables to a function of one variable. This process will be a key step in
nearly every problem in this next section.

\input{../leveledQuestions.tex}

\end{document}
